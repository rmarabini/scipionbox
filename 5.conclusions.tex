%\section{Discussion}
%[**TODO]: Discuss the obtained numbers in the result section
%[**REmove this section?]

\section{Conclusions}

The new generation of high-end cryo electron microscopes and direct electron detectors are behind the recent increase in cryo-EM  popularity. However, the run and maintenance costs of these new microscopes are high and can only be justified by keeping the microscopes up and running 24 hours a day. Therefore, microscopes tend to belong to centralized facilities where users are allocated microscope time (shifts) to perform a variety of tasks. Available shifts are a scarce commodity and therefore users (and staff) need to be able to monitor the quality of the acquisition process and obtain, as soon as possible, a first glimpse of the reconstructed specimen in order to provide an early detection of any issue. 

 Workflow systems have been explored and used extensively in bioinformatics but there is relatively little progress in that direction in the cryo-EM community. \scipion has been designed following a strict workflow paradigm. This approach is specially useful for tasks that need to go from a given start (movies) to a defined end (3D map), but where there are many different paths (algorithms) and steps to get there.
 
 In a workflow approach, users (including microscope staff) can reasonably be expected to edit the paths. They can potentially play with the pipeline in ways that aren't possible when having a programmer write code for them. Furthermore, not having the processing path hardcoded in the program means you don't have to recompile, so changes can be quickly tested   even at run time. The \scipion workflow approach provides provenance, that is,  for any created object (micrograph, CTFs, class, etc) \scipion stores metadata such as which programs, which parameters, and which exact path was followed to create any output data.  In this way,  real tracking and reproducibility are provided. 

 \scipion stream capabilities can handle large and continuously produced volumes of data. The benefits of streaming are clear: (1) it helps to process data as soon as it is generated,  providing almost instant results on data quality; (2) it saves user processing time and resources at her home institutions since part of the processing is done while data is being acquired; and (3) 
 it further reduces the overall experimental time by allowing processing the data while it is transferred to the final user laboratory. 
 
 \scipion has been deployed in cloud and there is a public image, with all the required preinstalled cryoEM software,  available at Amazon WebServices EC2 and EGI federated clouds. These images make use of \scipion streaming capabilities, that is, they are able to process data as soon as it has been transfered to an Amazon storage unit, and can be used to handle  production peaks.
 
 
\scipion is provided 
freely as open source software. Online documentation
describing \scipion download and installation is available
at \url{http://scipion.cnb.csic.es/m/download_form/} and \url{https://github.com/I2PC/scipion/wiki/How-to-Install}, respectively. Cloud installation instruction are available at \url{https://github.com/I2PC/scipion/wiki/Scipion-in-the-Cloud}. Other public software mentioned in this article as \emadmin, \emph{wizard session} and the different scripts used at eBIC are available at \url{https://github.com/rmarabini/webservices/tree/master/EMadmin}, \url{\url{https://github.com/delarosatrevin/scipion-session}} and \url{https://github.com/I2PC/scipion/tree/release-1.1-headless-devel} respectively.
If you would like to have more information about how to install \scipion or you have suggestions on how to improve it, contact us at scipion@cnb.csic.es.

% to automatically process user data within a few minutes of data collection at a rate of a few terabytes per day.

%In order to extract useful results from the image data, images must be processed using software which is complex, often lacking intuitive user interfaces and documentation




