\section{\scipion Setup in the Different Facilities}

The data flow created by an electron microscope requires a careful design of the IT infrastructures.  Different facilities has adopted different solutions and therefore, 
[**JMRT I don't really like how this sentence sounds. It suggest that Scipion installations are very different among facilities and this is not True, even if the initial wizard is or how folders are structured] how \scipion  access data and interchange information is very facility dependent. In this section we discuss  the different solutions applied in the different facilities in which \scipion is running.

So far, \scipion is being used routinely in four Cryo-EM facilities: \emph{Science for life Laboratory} (SciLifeLab, \url{http://www.scilifelab.se}),  \emph{European Synchrotron Radiation Facility} (ESRF, \url{http://www.esrf.eu/}), \emph{Diamond Light Source Synchrotron} (Diamond, \url{http://www.diamond.ac.uk}) and \emph{National Center for Biotechnology} (CNB, \url{http://www.cnb.csic.es}). \scipion original design satisfied the requirements of the CNB, in which \scipion automatically fetches newly recorded movies from a network mounted disk. After this first installation, \scipion was deployed on SciLifeLab (a centralized facility located at Stockholm) where a rigorous booking system is followed. There, \scipion was adapted to interface with this booking platform in order to produce better reports. Lately,  \scipion has been installed in a large synchrotron  such as ESRF and Diamond  where projects are handled by ISPyB (a customized laboratory management information system -LIMBS- \citep{Delageniere2011}) and data are saved on a distributed file system that heavily penalizes access to lists of files. In this environment, \scipion needs to constantly interchange information with ISPyB, and it cannot longer check if there is a new file in a particular storage device but must wait to be informed by the system that a new file has been created. All these particularities, that will be described in detail in the following, could only be handled thanks to a careful design that prioritizes flexibility at several levels. 


%\scipion is a very flexible framework that allows to create many different workflows and to choose among several algorithms at each workflow-step. In this work we do not recommend a particular workflow  over others since we realize that different specimens, microscopes or facilities may have different requirements. Nevertheless, since different use cases may clarify \scipionbox capabilities, in the following we describe the pipeline offered by default to the users of three different facilities where \scipion has been installed.

\subsection{CNB}

This section in divided in two parts. In the first one we discuss the network setup that allow us to move the data from the microscope to the user laboratory. Then  we comment on an additional software (called  \emadmin) that connects \scipion with the microscope acquisition 
software (EPU) and records the facility activity.

\subsubsection{Network Setup}

At the beginning CNB networking setup was a quite straight forward one, movies where stored in the  same machine that ran \scipion and processed data was saved to external USB hard disks by this same machine. The situation changed when the new Falcon-III detector was installed. After that, data was produced at sustained speeds as high as 90 MB/sec (when acquiring four movies by hole in lineal mode) and the main problem became to move smoothly the data through the entire pipeline from the microscope to the final user laboratory. In the following we describe such a pipeline.

From a network point of view there are three different levels of accessibility: (1) a private network that cannot be accessed by regular users, (2) the CNB LAN (local area network)  that can be accessed by any user which works inside the hosting  institution and (3) finally, there is a single computer that may be accessed from outside the institution and therefore, it can see the WAN (wide area network).

Data sets are recorded using EPU. The microscope control system has a 70 TB NAS (Network-attached storage) unit for saving movies as they are obtained (we will refer to this NAS as microscope-NAS). This storage unit is shared through a Samba share and connected to the private network using a 10Gb connexion. \scipion is executed on a Linux server which has two 1Gb NICs (network interface controllers) that connect it to the private and the LAN networks respectively (we will refer to this machine as \scipionbox). \scipionbox has 32 CPUs at 2.40GHz, 62 Gb of RAM memory and 2 \textit{Quadro M4000} GPUs. Finally, a second NAS with 4 1Gb NICs is linked to the private, LAN and WAN networks (we will refer to this second NAS as output-NAS).

The data flow is a follows: (a) the microscope system stores movies in the microscope-NAS, (b) when executing alignment-movie protocols \scipionbox will temporary read the movies from 
the microscope-NAS but will not store them locally, (c) each hour, output-NAS will connect to \scipionbox and copy or update the last \scipion project, (d) at the same time output-NAS may be either executing a script to incrementally backup the \scipion project and the movies to a USB disk or serve this information through the Internet and, (e) in parallel, \scipionbox transfers to  output-NAS different html reports that may be checked by users or staff with a HTML browser (see \url{http://nolan.cnb.csic.es/scipionbox}).

Although we try to discourage the use of external USB hard disks versus network as way of retrieving the raw and processing data, this is the preferred option for more of our users.
USB 3.0 drives can theoretically handle data flows of 300 MB/s, but our experience is that, in this busy environment, many of them  have trouble exceeding 50 MB/s. In most cases, 50 MB/s is enough to transfer the data before the following microscope shift starts. Nevertheless, real-world is tough and it happens that two, or even three,  users try to download their data simultaneously. Fortunately, output-NAS works as front-end and isolates the data acquiring and processing infrastructure from the user downloading. Network users access to output-NAS using a chrooted common account. This account only allows rsync connections and make impossible to retrieve data if the name of the project to be downloaded is unknown. In order to speed up transfer weak ciphers (as \emph{arcfour}) has been activated in output-NAS.     

\subsubsection{\emadmin}

\emadmin was born as a script that created a normalized tree of directories  in which to store the movies acqured by the microscope and help users to choose between a few predetermined image processing workflows. At present it was grown in a client-server application that in addition to the old script capacities it has incorporate the following functions:

\begin{enumerate}
 \item Transparent addition of new workflows. \scipion is able to export any executed workflow as a json file. \emadmin can import this files as use them to create new predetermined workflows.
 \item Automatic creation of pdf user's report containing: microscope acquisition parameters and histograms summarizing data resolution and  defocus. 
 \item The above mentioned information is stored in a database for further analysis. By default the system plots the  average resolution and astigmatism of the different acquisitions versus time
 \item Remote management. \emadmin is a client server application, the client maybe executed in a machine different from the one that executes \scipion
 \item Automatic backup of the \scipion project
 \item \scipion is able to create live html reports on the image acquisition process. \emadmin keeps track of all of them and facilitates the access to this information
\end{enumerate}

\emadmin is free software available at github repository (\url{https://github.com/rmarabini/webservices/tree/master/EMadmin}). You are welcomed to try it but the software is provided as it is and no support or maintenance service is offered. 



\subsection{Case 2: Scipion at the Swedish National Cryo-EM Facility}

The Swedish National Cryo-EM Facility offers access to state-of-the-art equipment and expertise in single particle cryo-EM and cryo electron tomography (cryo-ET). The Facility has two nodes: at \scilifelab in Stockholm and at Ume\aa\ University. \scilifelab in Stockholm offers single-particle service with a 
Talos Arctica for sample optimisation and a Titan Krios for high-resolution data collection. The Ume\aa\ node is expected to become operational during 2018, and will offer cryo-ET with a Titan Krios and a Scios DualBeam SEM.

The facilities can be accessed by Swedish researchers through 
a peer-reviewed process. Applications can be submitted through an application portal (\url{https://cryoem.scilifelab.se/}). % and will be evaluated once every three months based on their scientific merit and technical feasibility by a national Project Evaluation Committee. %On the other side, some time of the facility is reserved for internal research groups at Stockholm University. 
Internally, an online booking system called \emph{Booked Scheduler} centralizes the reservations of the microscopes and other instruments. 

At the beginning of 2016, the Cryo-EM facility at \scilifelab became one of the early adopters of \scipion streaming processing. Although there already were using home-made scripts to perform motion correction (with motioncor2) and CTF estimation (with Gctf), the microscope operators recognized the importance of a more general
framework. In particular, they were interested in the possibility of modifying easily the processing workflow as well as provide users and staff with  graphical tools for data analysis and quick feedback regarding the data collection.
Although initially the same setup script used at
CNB was adopted, a new one was developed to fully satisfy the facility
requirements. In the following sections we will briefly describe the computational infrastructure at the \scilifelab node and the implementation of a Session Wizard to automate the initial setup while fetching information from other sources external to \scipion. 
%[** thalos falcon III, titan k2 and falcon III]

\subsubsection{Network Setup and IT Infrastructure}
A storage and pre-processing server (the staging
server) constitutes the core of the SciLifeLab computational infrastructure.
 This machine has roughly 200 TB of storage (4 ZFS RAIDZ2 pools with 11 HDDs each), 2 NVIDIA GeForce GTX 1070 GPUs, 2 Intel Xeon E5-2630v4 CPUs (10 cores, 2.2 GHz, HT), 384 GB RAM, and a dual-port 10 GbE network card. The storage pool is exported via NFS and Samba. The microscope computers can thus write data directly there and other machines can access it.
 
% The storage pool is exported via NFS and Samba, so the microscope computers can write data there directly and other machines can access it.
%Unfortunately, the K2 [** Jose Miguel, what is K2?, do you have 2 cameras? in the same microscope or in different ones, this is confusing, comment when ] computer writes data to a local SSD RAID while the Falcon-III does it to a dedicated storage server. %, in order to avoid interruption of data collection due to network issues. 

To avoid interruption of data collection due to network issues each microscope has its own dedicated storage, that in the case of the Titan microscope is double since it may be connected to two different cameras, a K2 and a Falcon III. Hence a script to continuously move the produced data from the adequate storage to the staging server has been created. 

%In this way, after starting the data collection, users launch a script that continuously moves the produced data from the adequate storage to the staging server.

%ata on the staging server, in an organized folder structure, we
%can run pre-processing directly on the server.] If data processing is started soon
%after data collection the incoming data file should be in RAM-cache and  no
%disk access in needed, which is important with performance in mind. 
After processing, users may  copy their data to external USB drives from two workstations in the microscope room that have access to the storage pool via NFS.
There is also a dedicated download server which has access to the NFS export. This download server is accessible from  the Internet and allows downloads via rsync (over SSH) or SFTP. 

All these computers are connected to a 10 GbE network switch. The only machine accessible from the Internet is the data download server, which has a dual-port 10 GbE network adapter (one used for the internal network, one for the external network). %Both the K2 computer and the Falcon-III dedicated storage server had one unused 10GbE interface that could be used to connect them to our private network, without having to interfere with the default setup.

\subsubsection{Additional Software Developments: \emph{Session Wizard}}
In order to interface \scipion with the microscope infrastructure an upgraded version of the original script developed at CNB was used. Unfortunately, it was difficult to reuse the existing code due to a different folder structure and, more important, a different approach to user and microscope management. %To address the new set of requirements, a new wizard is under development at \scilifelab. For this new wizard, a basic data model was designed representing the current processing and data organization at \scilifelab. %The model is composed by the following entities:

%[** quitar modelo de datos]
%\begin{itemize}
%\setlength\itemsep{0em}
% \item \textit{User}: represents all persons that can book microscope time or are registered in the Application Portal 
% related to national projects.
% \item \textit{Reservation}:  is a resource (e.g, Krios, Talos, Vitrobot, etc) booking for a given period of time.
% \item \textit{Order}:  is a given national project with an unique identifier in the Application Portal
% \item \textit{Project}:  can be a either a national project (related to a given Order) or an internal project.
% \item \textit{Session}:  stores information about the usage of the micrograph for a given day.
%\end{itemize}

When a user starts a data collection, the wizard is executed using the command \textit{session-titan} (or \textit{session-talos}). The wizard determines which user has booked the microscope by fetching today's booking information from \emph{Booked Scheduler} through their web API. From the reservation information the wizard will also detect if it is an internal or a national facility project. 
In case of the latter, additional information will be retrieved from the Application Portal (such as principal investigator, project code, contact person or invoice address). After gathering all this information the wizard stores it, creates the processing environment and launches \scipion. %For example, the movies-import protocol will already point to the location where the files will be written for this session. Moreover, taking into account which microscope-camera is being  used, the files pattern can be inferred. 

The current implementation of the wizard stores each session in a simple database file. This database, together with extra information from the Booking System and the Application Portal, is used to generate invoices and reports for a given period. %The generation of a session report is under development, that will provide a summary and useful information for users to access the acquired data. 
%[** this last sentence is not very informative]

The described session wizard can be found at  \url{https://github.com/delarosatrevin/scipion-session}. You are welcomed to download and customize it to meet your organization specific requirements. \emph{Booked Scheduler} is available at (\url{https://www.bookedscheduler.com/})

%existing data model represents the specify usage at the \scilifelab facility and there are not immediate plans to make it more general and easy to reuse. Nonetheless, the model is quite simple and should to be difficult to modify it to fit other requirements. 




\subsection{Case 3: Scipion at eBIC facility at Diamond Light Source}

Diamond Light Source is the UK National Synchrotron User Facility.  As such, it provides access to large-scale experimental facilities to the UK and worldwide scientific communities to conduct experiments that are not possible in their home laboratories.  As a consequence of this national role, Diamond has significant experience at managing many visiting scientists and user experiments, and the associated data and post-processing.  %In the area of Macromolecular Crystallography, this has become a highly automated process which allows users to visualise electron density maps within minutes of the experiment being completed (REF. Aller P et al Methods Mol Biol 2015)
%The recent ``Revolution Resolution'' (REF. Kulbrandt W, Science 2014) in cryoEM has transformed the impact and demand of cryoEM in Structural Biology. However, the instrumentation and running costs of the latest generation of EM necessary for high-end cryoEM is beyond scope of all but the best-funded home laboratories. Due to the huge demand for high-end microscope access, 
The electron Bio-Imaging Centre (eBIC) was set up at Diamond~\citep{diamond2017}. The centre houses four Thermofisher Titan Krios, a Talos Arctica and a Scios dual-beam. All of the transmission electron microscopes are equipped with counting-mode direct electron detectors and Volta phase plates. 

Diamond has thousands of users every year, both academic and commercial, and it is vital that their data is kept secure and confidential. A user management system has been developed which allocates each user with a unique federal identity (IDs), and associates those user IDs with the experiments in which they partake in. Visits are structured such that all raw and processed data collected during an experiment are kept in one directory, and only the associated visitors have read and write permissions in that directory. Users may access their data through ISPyB \citep{ispb2011},  a Laboratory Information Management System (LIMS) combining sample tracking and experiment reporting  %The link between the user and their visits enables data management such as archiving data to a long-term tape storage system, or the Laboratory Information Management System ISPyB (REF. Delageniere et al Bioinformatics 2011). In contrast to the CNB setup, Diamond has integrated the microscopes in much the same fashion to that of the beamlines at the facility, as is shown in figure X
\subsubsection{Network Setup and IT Infrastructure}

Data is collected onto a local machine %in a similar way to the CNB setup, but 
and then immediately moved via 10Gbps Ethernet to the associated visit directory in the centralised storage, a multi Petabyte GPFS High Performance Parallel file system.  Once the raw data is present on the GPFS system, high speed interconnects to the central cluster enable rapid data processing. %Services such as these are shared amongst the whole facility allowing for flexibility in their deployment and generally allowing better service.

Once the data has reached the central file system, it is available to be processed by Scipion.  This process is facilitated by the use of a Scipion-workflow-template which enable novice users to trigger semi-customised on the fly processing. This template consists of a small UI (figure XX) which allows the user to put in the specific values pertinent to their experiment, namely the session ID (or visit number), which microscope they are using, and several other key data collection parameters.  Once set this makes use of a global template for Scipion as set up by the eBIC team, and substitutes the appropriate information into it, and then saves it to disk.

Following this Diamond makes use of a central ActiveMQ [** what is this? a queue system] service which is triggered upon the template file being available in the visit directory, and runs a headless (i.e. without UI) version of Scipion which takes the template and executes in streaming mode.  THIS SHOULD PROBABLY INCLUDE SOMETHING ABOUT THE DETECTION OF NEW FILES BY SOCKET MESSAGE, BUT I DON’T KNOW HOW THIS WORKS. Me neither, I thought it polled the filesystem?


To monitor the progress of the automatic processes, the Scipion pipeline at Diamond also includes custom nodes which allow it to interface with ISPyB.  This provides the users with a web interface showing general metadata about their experiment as well as new real-time updates on steps of the process and relevant output such as motion corrected images, drift values and CTF images (Figure XXX)


\subsection{What will do in future}
%Particle picking
%Extract particle
%2D classify (either classes or identify outliers)
%Initial model
%More complex reports
[** this section may disappear] 


