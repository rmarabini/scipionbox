\section{\scipion Setup at Different Facilities}

The data flow created by an electron microscope requires a careful design of the IT infrastructures.  Different facilities have adopted different solutions and therefore, how \scipion  access data and interchange information may require minor adjustments. In this section we discuss  the different solutions applied in the different facilities in which \scipion is running.

So far, \scipion is being used routinely in seven Cryo-EM facilities:
\begin{itemize}
 \itemsep0em 

 %\item \emph{Ume\aa\ Core Facility for Electron Microscopy} (UCEM, \url{http://www.kbc.umu.se/english/ucem})[** Jose Miguel can you verify this?] 
 \item \emph{The Swedish National Cryo-EM Facility}. The Facility has two nodes: at \scilifelab in Stockholm and at Ume\aa\ University (\url{http://www.scilifelab.se}, \url{http://www.kbc.umu.se/english/ucem/cryo-em/})
 \item \emph{ESRF Cryo-Electron Microscope} (\url{http://www.esrf.eu/home/UsersAndScience/Experiments/MX/About_our_beamlines/CM01.html})
 \item \emph{eBIC at Diamond Light Source} ( \url{http://www.diamond.ac.uk/Science/Integrated-facilities/eBIC.html})
 \item \emph{CNB CryoEM Service} (\url{http://www.cnb.csic.es/index.php/en/research/core-facilities/microscopia-crioelectronica})
 \item \emph{Molecular Microscopy Consortium - NIH} (\url{https://www.niehs.nih.gov/research/atniehs/facilities/mmc/index.cfm})
 \item \emph{National Cancer Institute - NIH} (\url{https://www.cancer.gov/research/resources/cryoem})
 
\end{itemize}
and it is being considered in many more.

\scipion original design satisfied the requirements of the CNB, in which \scipion automatically fetches newly recorded movies from a network mounted disk. After this first installation, \scipion was deployed on SciLifeLab (a centralized facility located at Stockholm) where a rigorous booking system is followed. There, \scipion was adapted to interface with this booking platform in order to produce better reports. Lately,  \scipion has been installed in  large synchrotrons  such as ESRF and Diamond  where projects are handled by ISPyB (a customized laboratory information management system -LIMS- \citep{Delageniere2011:ispb}) and data are saved on a distributed file system. % that heavily penalizes access to lists of files. 
In this environment, \scipion needs to constantly interchange information with ISPyB and with other applications through a message queue system called \emph{ActiveQ}.% and it cannot longer check if there is a new file in a particular storage device but must wait to be informed by the system that a new file has been created. 

In the following, through three use cases, we describe in detail how \scipion has been adapted to the different environments.

%All these particularities, that will be described in detail in the following, could only be handled thanks to a careful design that prioritizes flexibility at several levels. 

%\scipion is a very flexible framework that allows to create many different workflows and to choose among several algorithms at each workflow-step. In this work we do not recommend a particular workflow  over others since we realize that different specimens, microscopes or facilities may have different requirements. Nevertheless, since different use cases may clarify \scipionbox capabilities, in the following we describe the pipeline offered by default to the users of three different facilities where \scipion has been installed.

\subsection{CNB}

We are going to structure this section in three pats. In the first one we discuss the network setup that allow us to move the data from the microscope to the user laboratory. Then we present the standard image processing workflow and finally we comment on the additional software (different from \scipion) used in process data.

\subsubsection{Networking}

At the beginning CNB networking setup was a quite straight forward one, movies where stored in the  same machine that ran \scipion and processed data was saved to external USB hard disks by this same machine. The situation changed when the new Falcon-III detector was installed. After that, data was produced at sustained speeds as high as 90 MB/sec (when acquiring four movies by hole in lineal mode) and the main problem became to move smoothly the data through the entire pipeline from the microscope to the final user laboratory. In the following we describe such a pipeline.

From a network point of view there are three different levels of accessibility: (1) a private network that cannot be accessed by regular users, (2) the CNB LAN (local area network)  that can be accessed by any user which works inside the hosting  institution and (3) finally, there is a single computer that may be accessed from outside the institution and therefore, it can see the WAN (wide area network).

Data sets are recorded using EPU. The microscope control system has a 70 TB NAS (Network-attached storage) unit for saving movies as they are obtained (we will refer to this NAS as microscope-NAS). This storage unit is shared through a Samba share and connected to the private network using a 10Gb connexion. \scipion is executed on a Linux server which has two 1Gb NICs (network interface controllers) that connect it to the private and the LAN networks respectively (we will refer to this machine as \scipionbox). \scipionbox has 32 CPUs at 2.40GHz, 62 Gb of RAM memory and 2 \textit{Quadro M4000} GPUs. Finally, a second NAS with 4 1Gb NICs is linked to the private, LAN and WAN networks (we will refer to this second NAS as output-NAS).

The data flow is a follows: (a) the microscope system stores movies in the microscope-NAS, (b) when executing alignment-movie protocols \scipionbox will temporary read the movies from 
the microscope-NAS but will not store them locally, (c) each hour, output-NAS will connect to \scipionbox and copy or update the last \scipion project, (d) at the same time output-NAS may be either executing a script to incrementally backup the \scipion project and the movies to a USB disk or serve this information through the Internet and (e) in parallel \scipionbox transfers to  output-NAS different html reports that may be checked by users or staff with a HTML browser (see \url{http://nolan.cnb.csic.es/scipionbox}).

Although we try to discourage the use of external USB hard disks versus network. At present, this is the preferred option for more of our users.  USB 3.0 drives can theoretically handle data flows of 300 MB/s, but our experience is that, in this busy environment, many of them  have trouble exceeding 50 MB/s. In the ideal situation both raw and processed data are transfered as soon as available. In this way, access to the servers seldom overlaps between two users and once a microscope shift ends, users do not need to wait for their data to be transferred. fggdfgdsgdsfgdf. Comment on ssh cifers

comment on iperf and smbdata



\subsubsection{EMadmin}


\subsection{Swedish National CryoEM Facility}

\begin{verbatim}
We need to cover here differences with CNB setup
 CNB is a stand alone center. It does not work as a real National Facility
 SNC already had an application portal and Scipion needs to interact
 with it. 
 - initial script
 - same as CNB until session end
 - when acquisistion done contaCT PORTAL AND UPLOAD REPORT
\end{verbatim}



\subsection{Diamond Light Source Facility??? or eBIC???}

\begin{verbatim}
 
We need to cover here differences with CNB setup

 DEVELOPER RELATED
 * Describe infrastructure in diamond: file system (distributed, user permissions),
                                       no graphical output, queues, 
 * Describe Scipion modification:      import (detect new files by socket message)
                                       monitor, report to ISPyB 
                                       scipion headless
                                       execution as script

USER RELATED
 * how users interact with scipion?
 * how data is visualized?
\end{verbatim}






