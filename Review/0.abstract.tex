\begin{abstract}
  
Three dimensional electron microscopy is becoming a very data-intensive field in which vast amounts of experimental images are acquired at high speed. To manage such large-scale projects, we had previously developed  a modular workflow system called \scipion \citep{delarosatrevin2016}. We present here a major extension of \scipion that allows processing of EM images while the data is being acquired. This approach helps to detect problems at early stages, saves computing time and provides users with a detailed evaluation of the data quality before the acquisition is finished. %Different EM facilities have 
%implemented their stream processing solution using custom-made scripts, but this script based approach is not flexible and present a slow response time to new feature requests. The advantages of our proposed solution are: (1) flexibility: users can customize their workflows from a  large collection of preloaded algorithms; (2) repeatability and traceability: workflows are stored and can be re-executed later, either with the same data or with a different data set; (3) monitorization: there are utilities that constantly check how the execution of the algorithms is going on and inform about possible anomalies; (4) reports: we have developed several graphical interfaces that are updated periodically producing a summary (e.g, CTF defocus values, system load, etc). This summary may be generated in HTML format and  published through a public web-site to provide access for external users; (5) backup: both local and remote backup of the data and image processing is provided. 
At present, \scipion has been deployed and is in production mode in seven Cryo-EM facilities throughout the world.%, and it is being considered in many more.

\textbf{Keywords:} electron microscopy, streaming, image processing, live processing, high throughput, \scipion

\end{abstract}
