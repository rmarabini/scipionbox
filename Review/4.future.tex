\section{Future Developments and Usage}

\subsection{Future Developments}
Currently, \scipion includes the appropriate code (python wrappers) to talk to the integrated EM packages (Xmipp, Eman, Relion, ...) at the time of the release. Therefore, an update of any of the integrated EM packages immediately after the \scipion release will not be available until next \scipion release (we are aiming for one major release per year). To decouple EM-package releases from \scipion releases, we are working on making the wrappers for EM packages  independent from \scipion. Plans for ensuring this independence include: (1) wrapper installation through pip (\url{https://pip.pypa.io/en/stable/}), (2) reimplementation of the wrappers as plug-ins that will be automatically detected and added to the application menu and (3) develop of standalone IO  and visualization libraries that make \scipion fully independent from other EM packages.

Additionally, in collaboration with the Xmipp team, we are pushing the streaming workflow with the aim of obtaining a first initial volume. We are also adding structure modeling capabilities, by integrating some functionality from programs such as Coot \citep{emsley2010:coot}, Refmac \citep{Murshudov1997:refmac}, etc. 

As part of the EOSCPilot project (\url{https://eoscpilot.eu/cryoem}), we are also improving  how \scipion exports workflows in order to ease reporting of the work done. The goal is to write a workflow file that fully describes the image processing steps, so that if the same input data is provided to the workflow the same results should be obtained. In this way we want to make  the data produced by \scipion more compliant with FAIR guidelines given by the \citet{eu2016:fair}. This workflow file could go with the raw data acquired by CryoEM facilities and deposited at common EM databases such as EMPIAR or EMDB \citep{Patwardhan2016:databasesEM}. To facilitate the visualization of the workflow file in any of these databases, we are  developing a  webcomponent (\scipion \emph{workflow viewer}) that will easily allow these repositories to visualize the workflow on their web pages (\url{https://github.com/I2PC/web-workflow-viewer}). Finally, we are implementing a workflow repository. This repository will contain ready-to-use workflows that support a range of use cases.

\subsection{Usage}
In the context of free and open source projects it is difficult to create usage statistics that are accurate. But certainly, if we want \scipion to succeed, we need to 
know how frequent the different protocols are used. This is the reason why we have recently developed our \textit{Scipion Usage Data Collector} that monitors -if activated by the user- the usage of the different protocols and send -via HTTP protocol- the information to the developer's team. This information is anonymous and cannot be used to identify the original computer in which \scipion has been executed. 

Using the HTTP header it is possible to guess the country in which \scipion has been installed. With these information we have created the usage map and table available at URL \url{http://scipion.i2pc.es/report_protocols/scipionUsage/} and  \url{http://scipion.i2pc.es/report_protocols/protocolTable/}. The data corresponds on ``global'' \scipion usage rather than ``facility'' usage but there 
is a one to one relationship between the  countries with highest number of \scipion projects and the countries with \scipion based EM facilities. The only exception to the rule is Canada where two of the original \scipion developers are setting up a new facility (McGill University).

Note that, in a effort to report real use and skip test projects, the number of projects in the map refers to the number of non empty \scipion projects updated more than once. That is, 
\scipion projects that have been opened in at least two different days. Another constraint is that developers computers are filtered out.

