\documentclass[12, authoryear, preprint]{elsarticle}
\usepackage{graphicx}
\usepackage{amsmath}
\usepackage{amssymb}
\usepackage{natbib}
\usepackage[T1]{fontenc}
\usepackage[table]{xcolor} % Colors and tables
\usepackage{color, colortbl} % Define new colors and use in tables
\usepackage{setspace} % Set paragraph lines space
%\usepackage{tabu} % Other table package
\usepackage[hyphens]{url} % Other table package
\usepackage{listings}
\usepackage{rotating}
\usepackage[strict]{changepage}
\usepackage[caption=false]{subfig}
\usepackage{fancyhdr}
\usepackage{rotating}
\usepackage{xr}
\usepackage{xspace}
\usepackage{listings}

%\externaldocument[Supp-]{supplementary_material}

%\usepackage{lineno}
\usepackage[update,prepend]{epstopdf}
%\linenumbers
\definecolor{lightgreen}{rgb}{0.75,0.75,0.0}
%\usepackage{caption}
%\usepackage{endfloat}
%\DeclareCaptionFormat{empty}{}
%\captionsetup{format=empty}
\pagestyle{fancy}
\fancyhf{} % clear all header and footer fields
\fancyfoot[R]{\footnotesize  \thepage}
\renewcommand{\headrulewidth}{0pt}
\renewcommand{\footrulewidth}{0pt}

\def\scipionbox{\textit{ScipionBox}\xspace}
\def\mnas{\textit{microscope-NAS}\xspace}
\def\onas{\textit{Output-NAS}\xspace}
\def\hserver{\textit{HTML-Server}\xspace}
\def\scipion{\textit{Scipion}\xspace}
\def\emadmin{\textit{EMAdmin}\xspace}
\def\python{\textit{Python}\xspace}

\def\epu{\textit{EPU}\xspace}
\def\scilifelab{SciLifeLab\xspace}
\def\cnbcsic{CNB-CSIC\xspace}
\def\cnb{CNB\xspace}

\newcommand{\protocol}[1]{\texttt{#1}}

%%%%%%%%%%%%%%%%%\def\baselinestretch{2}
%\def\PARstart#1#2{#1#2} % if draft, disable it
\def\degree{\ensuremath{^\circ}}

%%%%%%%%%%%%%%%%%%%%%%%%%%%%%% Textclass specific LaTeX commands.
\newcommand{\lyxaddress}[1]{
   \par {\raggedright #1
   \vspace{1.4em}
   \noindent\par}
}
 
\newcommand{\itemph}[1]{\item{\emph{#1}}}
\newcommand{\itprog}[1]{\item{\texttt{#1}}:}
\newcommand{\refig}[1]{Fig. \ref{#1}}
\newcommand{\seefig}[1]{(see \refig{#1})}
\definecolor{grey}{rgb}{0.9,0.9,0.9}
\renewcommand{\arraystretch}{1.3}

\newcommand{\ffigure}[1]{Figure \ref{#1}}
\newcommand{\ttable}[1]{Table \ref{#1}}
\newcommand{\ffigures}{Figures}
\newcommand{\ttables}{Tables}



\lstset{ %
  language=XML,                % the language of the code
  basicstyle=\footnotesize{8pt},       % the size of the fonts that are used for the code
  %numbers=left,                   % where to put the line-numbers
  %numberstyle=\tiny\color{gray},  % the style that is used for the line-numbers
  %stepnumber=2,                   % the step between two line-numbers. If it's 1, each line 
                                  % will be numbered
  %numbersep=5pt,                  % how far the line-numbers are from the code
  backgroundcolor=\color{grey},      % choose the background color. You must add \usepackage{color}
  %showspaces=false,               % show spaces adding particular underscores
  %showstringspaces=false,         % underline spaces within strings
  %showtabs=false,                 % show tabs within strings adding particular underscores
  frame=single,                   % adds a frame around the code
  %rulecolor=\color{black},        % if not set, the frame-color may be changed on line-breaks within not-black text (e.g. commens (green here))
  %tabsize=2,                      % sets default tabsize to 2 spaces
  %captionpos=b,                   % sets the caption-position to bottom
  %breaklines=true,                % sets automatic line breaking
  %breakatwhitespace=false,        % sets if automatic breaks should only happen at whitespace
  %title=\lstname,                   % show the filename of files included with \lstinputlisting;
                                  % also try caption instead of title
  keywordstyle=\color{blue},          % keyword style
  %commentstyle=\color{dkgreen},       % comment style
  %stringstyle=\color{mauve},         % string literal style
  %escapeinside={\%*}{*)},            % if you want to add a comment within your code
  %morekeywords={*,...}               % if you want to add more keywords to the set
}

\begin{document}
\title{Using \scipion for Stream Image Processing at Cryo-EM Facilities}
\begin{frontmatter}

\author[canada]{J. G\'{o}mez-Blanco\tnoteref{first}}
\author[su1]{J.M. de la Rosa-Trev\'{i}n\tnoteref{first}}
\author[uam]{R.~Marabini\corref{corr}\tnoteref{first}}
\ead{roberto.marabini@uam.es}
\cortext[corr]{Corresponding author}
\cortext[first]{These authors  contributed equally to this work.}
\author[cnb]{L. del Cano}
\author[cnb]{A. Jim\'{e}nez}
\author[cnb]{M. Mart\'{i}nez}
\author[cnb]{R. Melero}
\author[cnb]{T. Majtner}
\author[cnb]{D. Maluenda}
\author[cnb]{J. Mota}
\author[cnb]{Y. Rancel}
\author[cnb]{E Ram\'{i}rez-Aportela}
\author[cnb]{J.L. Vilas}
\author[su1]{M. Carroni}
\author[su1]{S. Fleischmann}
\author[su1, su2]{E. Lindahl}
\author[ebic]{A.W. Ashton}
\author[ebic]{M. Basham}
\author[ebic]{D.K. Clare}
\author[ebic]{K. Savage}
\author[ebic]{C.A. Siebert}
\author[mrc]{G.G. Sharov}
\author[cnb]{C.O.S. Sorzano}
\author[cnb]{P. Conesa}
\author[cnb]{J.M. Carazo}


%xxxx$^{(1)}$, xxxxx$^{(1,3)}$}
%\maketitle
\address[canada]{Department of Anatomy and Cell Biology, McGill University, Montreal, Canada}
\address[cnb]{Biocomputing Unit, National Center for Biotechnology (CSIC), C/ Darwin, 3, Campus Universidad Aut\'{o}noma, 28049 Cantoblanco, Madrid, Spain.}
\address[su1]{Department of Biochemistry and Biophysics, Science for Life Laboratory, Stockholm University, Stockholm, Sweden} 
\address[uam]{Escuela Polit\'{e}cnica Superior, Universidad Aut\'{o}noma de Madrid, 28049 Cantoblanco, Madrid, Spain.}
\address[su2]{Swedish e-Science Research Center, KTH Royal Institute of Technology, Stockholm, Sweden}
\address[ebic]{Diamond Light Source, Harwell Science and Innovation Campus, Didcot OX11 0DE, United Kingdom}
\address[mrc]{Medical Research Council Laboratory of Molecular Biology, Francis Crick Avenue, Cambridge CB2 OQH, UK}


%This document provides a preliminary analysis of the data collected by the CTF Challenge (http://i2pc.cnb.csic.es/3dembenchmark/LoadHome.htm). Figures/tables/plots/etc are located at the document's end. 
%\setcounter{figure}{2}%I do not know why but figure counter starts with -1
%Escuela Polit\'{e}cnica Superior\\
%Universidad Aut\'{o}noma de Madrid\\
%28049 Cantoblanco\\
%Madrid\\
%Spain

\begin{abstract}
  
Three dimensional electron microscopy (3DEM) is becoming a very data-intensive field in which vast amounts of experimental data are acquired at high speed. To manage such large-scale image data  projects, we have previously described  a modular workflow system called \scipion. We present here the first main extension of \scipion that allows to process data in streaming. Streaming processing allows to overlap data acquisition with the first steps of digital image processing and therefore helps to: (1) detect problems at early stages, (2) save computing time and (3) provide users with a in deep evaluation of the data quality before the data acquisition is finished. This idea of processing the data while being acquired has been implemented in different laboratories mainly using custom-made scripts. The advantages of our \scipion solution are: (1) flexibility: users can customize their workflows from a  large collection of preloaded algorithms; (2) repeatability and traceability: workflows are stored and can be re-executed later, either with the same data or a different data set; (3) monitorization: there are special utilities that constantly check how the execution of the algorithms is going and inform about possible anomalies; (4) reports: we have developed several GUIs that are updated periodically and produce a graphical summary (e.g, CTF defocus values, system load, etc). This summary may be  generated in HTML format that can be easily published though a public web-site to provide access for external users; (5) backup: both local and remote backup of the data and image processing is provided. At present, \scipion has been installed and is in production mode in three facilities. And it is being considered in several more.

\textbf{Keywords:} electron microscopy

\end{abstract}

\end{frontmatter}

\section{Introduction}

3D electron microscopy (3DEM)  can provide rich information about structural characteristics of macromolecular complexes. The field is under a major transformation due to the arrival of better microscopes, new large area detectors and automatization \citep{kuhlbrandt2014a, Kuhlbrandt2014b}. These improvements make possible that a single microscope generates high quality data sets on the order of terabytes per day \citep{Saibil2015} while working for several days without interruption. Major challenges faced in the field are: (1) an efficient management of these huge datasets and their corresponding image processing workflows; (2) a raising interest from scientists outside the field who might lack the skills of a experienced microscopist and; (3) the need for an effective data and metadata transfer in an increasingly distributed and collaborative environment. %This situation strongly demands the application of techniques that are highly automated to allow high throughput. 

%We will focus in this article in the critical stage of the image data acquisition in 3DEM.  Our goal is  to further develop  (\scipion) such that it extends the functionality of the microscope data collecting programs. In this way, \scipion starts the processing of the movies as they are being acquired with the double aim of checking for possible data collection errors and execute a first user-tailored image processing workflow that provides users with an accurate measurement of the acquired data quality.

%Accounting for major government investment in increasingly costly electron microscopes requires a large user  base. 

Clearly, the future of projects that require high-end microscopes is through centralized microscopy facilities, in which automatic acquired data needs to be monitored by users and facility staff.  \scipion has been redesigned with this scenario in mind. % in which the early detection of issues may save a data acquisition shift which otherwise needs to be repeated after a new application that may be granted many months later. 
In this way, \scipion starts the processing of the movies as they are being acquired with the double aim of checking for possible data collection errors and execute a first user-tailored image processing workflow that provides users with an accurate measurement of the acquired data quality.

Many software packages have been developed to provide automated data collection, such as SerialEM \citep{Mastronarde2005}, Leginon \citep{Suloway2009}, UCSF-Tomography \citep{Zheng2007}, Tom Toolbox \citep{Nickell2005}, EPU \citep{EPU}, Latitude in Digital Micrograph \citep{Latitude}, EMMenu \citep{emmenu}, etc. However, the next step,  which is to provide image processing while  data are being collected, is not so well addressed. In fact, in most places this is accomplished through home made scripts \citep[i.e.][]{Pichkur2018}. %(i.e. \citet{Pichkur2018}).
Exceptions are UCSFImage4 \citep{Li2015} and Focus \citep{Biyani2017}. As we will show in this article, \scipion differs from these solutions in a larger flexibility both from the point of view of users and facility staff.

\section{Main Modifications introduced in  \scipion: Streaming, Monitors and Consensus}
\label{overall}

In the following we describe the main modifications introduced in the original \scipion version in order to facilitate its deployment in centralized facilities. 

The main change introduced in \scipion is \emph{stream processing}. Stream processing has been designed to analyze and react on real-time so that enables analysis of data as it is produced.
In this way a new level of parallelization has been reached since the  data generated by the microscope continuously flows through a chain of protocols. For example, as movies are imported to the system they are aligned and their CTF is estimated. In \scipion stable version users may find  the first steps in the image processing adapted to stream. These steps include: movie alignment, CTF estimation and particle picking. In \scipion developers version  several 2D classification algorithms are available.

\scipion stream implementation is based on periodic updates of the protocol input and output data during the image processing. Since \scipion runs on threads, it is possible to devote one thread to update these information while the rest of the CPU power is devoted to run the different protocols. The use of threads marks an important difference between running a script or using \scipion, since  each module is run independently creating a very flexible processing framework. For example, do you want to add a new protocol in your workflow while it is running? No problem, do it. \scipion allows users to modify  workflows on the fly without modifying a single code line. The modification will not interfere with the already executed steps and may take advantage from all already computed data. Did something go wrong and the microscope stopped? By default \scipion will stop after 24 hours with no microscope activity, but if you wish to continue your acquisition after a stop just press the continue button, there is no need to restart from the beginning. 

%\scipionbox uses the infrastructure provided by the Scipion project [Ref if not given already] that takes care of the creation and execution of workflows. These workflows may be imported from a set of predefined ones or created by the final users even if they have no programing skills.

%A typical \scipion workflow starts importing the microscope data, then the  beam-induced movement is corrected. For this task, \scipion allows you to choose between five different algorithms from three different labs (an exhaustive list containing the approximately 200 algorithms accessible through \scipion and ranked by usability is available at \url{http://calm-shelf-73264.herokuapp.com/report_protocols/protocolTable/}). This multiplicity of possibilities is available at each different processing step although, if found overwhelming, may be ignored by choosing one of the default prepackaged workflows.   

Since \scipion original publication we have developed many \emph{metaprotocols}, that is, 
protocols that either check the progress of other protocols or compare the results of equivalent protocols.

The first type of metaprotocols are called \emph{monitors} and are used to produce live analysis plots, generate reports or raise alerts when some problems are detected. A monitor example is the \emph{CTF-monitor}, that checks the computed defocus values for
each micrograph as they are generated. CTF-monitor may raise an alert if the defocus values are above or below certain thresholds and continuously generate HTML files so that users and staff may easily follow the data acquisition and processing in-house or remotely. An example of this reports may be seen at \url{http://nolan.cnb.csic.es/scipionbox/} [**ROB NOTE: select a nice one an complete URL]%At the project's end, a review of the processing history is produced. [** URL to report example?]
%This feature was very important since we knew that difference facilities will have different needs and criteria to be fulfilled.

The second set of metaprotocols are grouped under the name \emph{consensus}. For a given logical step (for example, particle picking) these protocols check if the datasets obtained from the same input data using different algorithm are consistent. Continuing with the particle picking example, \emph{consensus picking} will compare the particles selected for each one of the executed particle-picking algorithms and produce as output a particle set containing those particles selected by most of the algorithms.


Last but not least \scipion provides full data provenance. Data provenance is an important form of metadata that describes how a particular data set was generated by detailing: the steps, input data and parameters used in the computation. Data provenance guaranties reproducibility and traceability. %Since provenance is solved by Scipion and not Scipionbox we will not discuss here how it is been achieved. We just note that, before choosing which data has to be stored, it is necessary to define how these data have to be structured so that they can be later recovered and understood . %This abstract modeling of the data also simplifies the software interoperability offered by Scipion.


%Standardization
%User has access to all parameters used by algorithms (she may ignore them)
%Project can be imported in scipion without further ado.
%Gui vs programatico
%Export options?


 


\section{\scipion Setup at Different Facilities}

The data flow created by an electron microscope requires a careful design of the IT infrastructures.  Different facilities have adopted different solutions and therefore, how \scipion  access data and interchange information may require minor adjustments. In this section we discuss  the different solutions applied in the different facilities in which \scipion is running.

So far, \scipion is being used routinely in seven Cryo-EM facilities:
\begin{itemize}
 \itemsep0em 

 %\item \emph{Ume\aa\ Core Facility for Electron Microscopy} (UCEM, \url{http://www.kbc.umu.se/english/ucem})[** Jose Miguel can you verify this?] 
 \item \emph{The Swedish National Cryo-EM Facility}. The Facility has two nodes: at \scilifelab in Stockholm and at Ume\aa\ University (\url{http://www.scilifelab.se}, \url{http://www.kbc.umu.se/english/ucem/cryo-em/})
 \item \emph{ESRF Cryo-Electron Microscope} (\url{http://www.esrf.eu/home/UsersAndScience/Experiments/MX/About_our_beamlines/CM01.html})
 \item \emph{eBIC at Diamond Light Source} ( \url{http://www.diamond.ac.uk/Science/Integrated-facilities/eBIC.html})
 \item \emph{CNB CryoEM Service} (\url{http://www.cnb.csic.es/index.php/en/research/core-facilities/microscopia-crioelectronica})
 \item \emph{Molecular Microscopy Consortium - NIH} (\url{https://www.niehs.nih.gov/research/atniehs/facilities/mmc/index.cfm})
 \item \emph{National Cancer Institute - NIH} (\url{https://www.cancer.gov/research/resources/cryoem})
 
\end{itemize}
and it is being considered in many more.

\scipion original design satisfied the requirements of the CNB, in which \scipion automatically fetches newly recorded movies from a network mounted disk. After this first installation, \scipion was deployed on SciLifeLab (a centralized facility located at Stockholm) where a rigorous booking system is followed. There, \scipion was adapted to interface with this booking platform in order to produce better reports. Lately,  \scipion has been installed in  large synchrotrons  such as ESRF and Diamond  where projects are handled by ISPyB (a customized laboratory information management system -LIMS- \citep{Delageniere2011:ispb}) and data are saved on a distributed file system. % that heavily penalizes access to lists of files. 
In this environment, \scipion needs to constantly interchange information with ISPyB and with other applications through a message queue system called \emph{ActiveQ}.% and it cannot longer check if there is a new file in a particular storage device but must wait to be informed by the system that a new file has been created. 

In the following, through three use cases, we describe in detail how \scipion has been adapted to the different environments.

%All these particularities, that will be described in detail in the following, could only be handled thanks to a careful design that prioritizes flexibility at several levels. 

%\scipion is a very flexible framework that allows to create many different workflows and to choose among several algorithms at each workflow-step. In this work we do not recommend a particular workflow  over others since we realize that different specimens, microscopes or facilities may have different requirements. Nevertheless, since different use cases may clarify \scipionbox capabilities, in the following we describe the pipeline offered by default to the users of three different facilities where \scipion has been installed.

\subsection{CNB}

We are going to structure this section in three pats. In the first one we discuss the network setup that allow us to move the data from the microscope to the user laboratory. Then we present the standard image processing workflow and finally we comment on the additional software (different from \scipion) used in process data.

\subsubsection{Networking}

At the beginning CNB networking setup was a quite straight forward one, movies where stored in the  same machine that ran \scipion and processed data was saved to external USB hard disks by this same machine. The situation changed when the new Falcon-III detector was installed. After that, data was produced at sustained speeds as high as 90 MB/sec (when acquiring four movies by hole in lineal mode) and the main problem became to move smoothly the data through the entire pipeline from the microscope to the final user laboratory. In the following we describe such a pipeline.

From a network point of view there are three different levels of accessibility: (1) a private network that cannot be accessed by regular users, (2) the CNB LAN (local area network)  that can be accessed by any user which works inside the hosting  institution and (3) finally, there is a single computer that may be accessed from outside the institution and therefore, it can see the WAN (wide area network).

Data sets are recorded using EPU. The microscope control system has a 70 TB NAS (Network-attached storage) unit for saving movies as they are obtained (we will refer to this NAS as microscope-NAS). This storage unit is shared through a Samba share and connected to the private network using a 10Gb connexion. \scipion is executed on a Linux server which has two 1Gb NICs (network interface controllers) that connect it to the private and the LAN networks respectively (we will refer to this machine as \scipionbox). \scipionbox has 32 CPUs at 2.40GHz, 62 Gb of RAM memory and 2 \textit{Quadro M4000} GPUs. Finally, a second NAS with 4 1Gb NICs is linked to the private, LAN and WAN networks (we will refer to this second NAS as output-NAS).

The data flow is a follows: (a) the microscope system stores movies in the microscope-NAS, (b) when executing alignment-movie protocols \scipionbox will temporary read the movies from 
the microscope-NAS but will not store them locally, (c) each hour, output-NAS will connect to \scipionbox and copy or update the last \scipion project, (d) at the same time output-NAS may be either executing a script to incrementally backup the \scipion project and the movies to a USB disk or serve this information through the Internet and (e) in parallel \scipionbox transfers to  output-NAS different html reports that may be checked by users or staff with a HTML browser (see \url{http://nolan.cnb.csic.es/scipionbox}).

Although we try to discourage the use of external USB hard disks versus network. At present, this is the preferred option for more of our users.  USB 3.0 drives can theoretically handle data flows of 300 MB/s, but our experience is that, in this busy environment, many of them  have trouble exceeding 50 MB/s. In the ideal situation both raw and processed data are transfered as soon as available. In this way, access to the servers seldom overlaps between two users and once a microscope shift ends, users do not need to wait for their data to be transferred. fggdfgdsgdsfgdf. Comment on ssh cifers

comment on iperf and smbdata



\subsubsection{EMadmin}


\subsection{Swedish National CryoEM Facility}

\begin{verbatim}
We need to cover here differences with CNB setup
 CNB is a stand alone center. It does not work as a real National Facility
 SNC already had an application portal and Scipion needs to interact
 with it. 
 - initial script
 - same as CNB until session end
 - when acquisistion done contaCT PORTAL AND UPLOAD REPORT
\end{verbatim}



\subsection{Diamond Light Source Facility??? or eBIC???}

\begin{verbatim}
 
We need to cover here differences with CNB setup

 DEVELOPER RELATED
 * Describe infrastructure in diamond: file system (distributed, user permissions),
                                       no graphical output, queues, 
 * Describe Scipion modification:      import (detect new files by socket message)
                                       monitor, report to ISPyB 
                                       scipion headless
                                       execution as script

USER RELATED
 * how users interact with scipion?
 * how data is visualized?
\end{verbatim}







\subsection{Future Developments}

Currently, \scipion includes the appropriate code (python wrappers) to talk to the integrated EM packages (Xmipp, Eman, RELION, ...) at the time of the release. This implies that an update of any of the integrated EM packages immediately after the \scipion release, won't be available until next \scipion release (we are aiming for one release a year). To decouple EM-package releases from \scipion releases, we are working on making the wrappers for EM packages  independent from \scipion. Plans for ensuring this independence include: wrapper installation through pip (\url{https://pip.pypa.io/en/stable/}) and the reimplementation of the wrappers as plug-ins that will be automatically detected and added to the application menu. 

Additionally, in collaboration with the Xmipp team, we are pushing the streaming workflow with the aim to obtain a first initial volume. We are also adding structure modeling capabilities, by integration of some functionality from programs such as Coot \citep{emsley2010:coot}, Refmac \citep{Murshudov1997:refmac}, etc. 

As part of the EOSCPilot project (\url{https://eoscpilot.eu/cryoem}), we are also improving  how \scipion exports workflows in order to ease reporting of the work done. The goal is to write a workflow file that fully describes the image processing steps, so that if the same input data is provided to the workflow the same results should be obtained (making the data more compliant with FAIR guidelines given by the \citet{eu2016:fair}). This file could go with the raw data acquired by CryoEM facilities and deposited at common EM databases like EMPIAR or EMDB \citep{Patwardhan2016:databasesEM}. To facilitate the visualization of the workflow file in any of these databases, we are also developing a  webcomponent (\scipion \emph{workflow viewer}) that will easily allow these repositories to visualize the workflow on their web pages (\url{https://github.com/I2PC/web-workflow-viewer}). Finally, we are implementing a workflow repository. This repository will contain ready-to-use workflows that support a range of use cases.

\section{Discussion}
[**TODO]: Discuss the obtained numbers in the result section
[**REmove this section?]

\section{Conclusions}

The new generation of high-end cryo electron microscopes and direct electron detectors are behind the recent increase in cryo-EM  popularity. However, the run and maintenance costs of these new microscopes are high and can only be justified by keeping the microscopes up and running 24 hours a day. Therefore, microscopes tend to belong to centralized facilities where users are allocated microscope time (shifts) to perform a variety of tasks. Available shifts are a scarce commodity and therefore users (and staff) need to be able to monitor the quality of the acquisition process and obtain as soon as possible a first glimpse of the reconstructed specimen in order to provide an early detection of any issue. 

 Workflow systems have been explored and used extensively in bioinformatics but there is relatively little progress in that direction in the cryo-EM community. \scipion has been designed following a strict workflow paradigm. This approach is specially useful for tasks that need to go from a given start (movies) to a defined end (3D map) but there are many different paths (algorithms) and steps to get there.
 
 In a workflow approach, users (including microscope staff) can reasonably be expected to edit the paths. They can potentially play with the pipeline in ways that aren't possible when having a programmer write code for them. Furthermore, not having the processing path hardcoded in the program means you don't have to recompile, so changes can be quickly tested   even at run time. The \scipion workflow approach provides provenance, that is,  for any created object (micrograph, CTFs, class, etc) \scipion stores metadata such as which programs, which parameters, and which exact path was followed to create it.  In this way,  real tracking and reproducibility are provided. 

 \scipion stream capabilities can handle large and continuously produced volumes of data. The benefits of streaming are clear: (1) it helps to process data as soon as it is generated,  providing almost instant results on data quality; (2) it saves user processing time at their home institutions since part of the processing is done while data is being acquired; and (3) 
 it further reduces the overall experimental time by allowing processing the data while it is transferred to the final user laboratory. \scipion has been deployed in cloud and there is a  a public image, with all the required preinstalled cryoEM software,  available at Amazon WebServices EC2 and EGI federated clouds. These images make use of \scipion streaming capabilities, that is, they are able to process data as soon as it has been transfered to an Amazon storage unit, and can be used to handle  production peaks.
 
 
\scipion is provided 
freely as open source software. Online documentation
describing \scipion download and installation is available
at \url{http://scipion.cnb.csic.es/m/download_form/} and \url{https://github.com/I2PC/scipion/wiki/How-to-Install}, respectively. Cloud installation instruction are available at \url{https://github.com/I2PC/scipion/wiki/Scipion-in-the-Cloud}. Other public software mentioned in this article as \emadmin, XXXX and YYYY are available at \url{https://github.com/rmarabini/webservices/tree/master/EMadmin}, www eee respectively.
If you would like more information about how to install \scipion or you have suggestions on how to improve it contact us at scipion@cnb.csic.es.

% to automatically process user data within a few minutes of data collection at a rate of a few terabytes per day.

%In order to extract useful results from the image data, images must be processed using software which is complex, often lacking intuitive user interfaces and documentation



 
\section{Acknowledgments}

The authors would like to acknowledge economical support from:

The Spanish Ministry of Economy and Competitiveness through Grants BIO2013-44647-R, BIO2016-76400-R(AEI/FEDER, UE) and AEI/FEDER BFU 2016 74868P, the Comunidad Aut\'{o}noma de Madrid through Grant: S2017/BMD-3817, European Union (EU) and Horizon 2020 through grant CORBEL (INFRADEV-1-2014-1, Proposal: 654248). The ``Knut \& Alice Wallenberg Foundation'', and ``A Pilot Facility development grant from Science for Life Laboratory''. 

This work used the EGI Infrastructure and is co-funded by the EGI-Engage project (Horizon 2020) under Grant number 654142. European Union (EU) and Horizon 2020 through grant West-Life (EINFRA-2015-1, Proposal: 675858) European Union (EU) and Horizon 2020 through grant Elixir - EXCELERATE (INFRADEV-3-2015, Proposal: 676559) European Union (EU) and Horizon 2020 through grant iNEXT (INFRAIA-1-2014-2015, Proposal: 653706). The authors acknowledge the support and the use of resources of Instruct, a Landmark ESFRI project. This work has used the Cryo-EM Swedish National Facility funded by the Knut and Alice Wallenberg, Family Erling Persson and Kempe Foundations, SciLifeLab, Stockholm University and Ume\aa\ University.

We acknowledge Diamond for access and support of the Cryo-EM facilities at the UK national electron bio-imaging centre (eBIC), funded by the Wellcome Trust, MRC and BBSRC.

----------------

    AFTER THIS LINE I JUST SAVED SOME TEXT THAT MAY BE USEFUL IN THE FUTURE DISREGARD THIS MATERIAL

    --------------------
   


\section{Project stability and funding}


Scipion development is been done in the context of the Instruct Image Processing Cen-
ter (I2PC, http://i2pc.cnb.csic.es). Instruct is the European initiative for Strategic
Scientific Infrastructures (ESFRI) for Structural Biology. The I2PC started with the objective 
to provide efficient support for research projects in Europe demanding expertise in 
electron microscopy image processing. In order to support a large collection
of projects, the first activities of the I2PC were oriented to design and implement the
necessary software tools. 

At present the I2PC has this money and this people

ScipionBox is part of the Scipion package which is provided 
freely as open source software. Online documentation
describing Scipion download and installation is available
at \url{http://scipion.cnb.csic.es/m/download_form/} and \url{https://github.com/I2PC/scipion/wiki/How-to-Install}, respectively.

%It was recognized the need for a platform that could provide
%a friendly GUI on top of advanced image processing algorithms and, at the same time,
%assure standardization, traceability and reproducibility.
\section{Things to mention if possible}
\begin{itemize}
 \item Cloud
\end{itemize}

%\appendix

\section{References}
\bibliographystyle{elsarticle-harv}
\bibliography{large}

\clearpage
%\input{listOfFigures.tex}

\newpage
%tables
%\input{numberImagesDataSet.tex}
%%%%%\input{markersLegend.tex}
%\input{numberImagesUpload.tex}
%\newpage



\end{document}
