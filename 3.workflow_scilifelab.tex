
\subsection{Swedish National CryoEM Facility}

\begin{verbatim}
We need to cover here differences with CNB setup
 CNB is a stand alone center. It does not work as a real National Facility
 SNC already had an application portal and Scipion needs to interact
 with it. 
 - initial script
 - same as CNB until session end
 - when acquisistion done contaCT PORTAL AND UPLOAD REPORT
\end{verbatim}

The Swedish National Cryo-EM Facility offers access to state-of-the-art equipment and expertise 
in single particle cryo-EM and cryo electron tomography (cryo-ET). The Facility has two nodes: at SciLifeLab 
in Stockholm and at Umeå University. SciLifeLab in Stockholm offers single-particle service with a 
Talos Arctica for sample optimisation and a Titan Krios for high-resolution data collection. 
The Umeå node is expected to become operational during 2018, and will offer cryo-ET with a 
Titan Krios and a Scios DualBeam SEM.

The facilities can be accessed by Swedish researchers through a peer-reviewed routine. Applications can be 
submitted through an Application Portal (reference???) and will be evaluated once every three months based on their 
scientific merit and technical feasibility by a national Project Evaluation Committee. On the other side, 
some time of the facility is reserved for internal research groups at Stockholm University. Internally, the allocation
of the microscope (and other instruments) time is managed by a Booking System (reference???). 


