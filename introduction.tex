
\section{Introduction}

3D electron microscopy (3DEM)  can provide rich information about structural characteristics of macromolecular complexes. The field is under a major transformation due to the arrival of better microscopes [**REF], new large area detectors [**REFS ddd] and automatization [**REFs to data colection software as EPU, leginon, serialEm,etc...]  which makes possible that a single microscope generates high quality data sets on the order of terabytes per day [**REF] while working for several days without interruption. Major challenges faced in the field are: (1) an efficient management of these huge datasets and their corresponding image processing workflows; (2) a raising interest from scientists outside the field who might lack the skills of experienced microscopist and; (3) an the need for an effective transference of data and associated metadata in an increasingly distributed and collaborative environment. This situation places techniques that are automated and allow higher throughput in high demand. 

We will focus in this article in the critical stage of the image data acquisition in 3DEM.  Our goal is  to develop software (\scipionbox) able to extend the functionality of the microscope data collecting programs. In this way, \scipionbox starts the processing of the movies as they are being acquired with the double aim of checking for possible data collection errors and execute a first user-tailored image processing workflow that provides users with an accurate measurement of the acquired data quality.

Accounting for major government investment in increasingly costly electron microscopes requires a large user  base. Clearly, the future of projects that require this high-end microscopes is through centralized microscopy facilities, in which automatic acquired data needs to be monitored by users and facility staff. \scipionbox has been designed with this scenario in mind in which the early detection of issues may save the data acquisition shift which otherwise need to be repeated after a new application that may be allowed many months later.

Many software packages have been developed to provide automated data collection, such as SerialEM (Mastronarde, 2005), Leginon (Suloway et al., 2005), UCSF-Tomography (Zheng et al., 2007), Tom Toolbox (Nickell et al.2005), EPU (Thermo Fisher Co.), Latitude in Digital Micrograph (Gatan Co.), or EMMenu
(TVIPS, Germany), etc. However, the next step,  which is to provide image processing while  data are being collected, is not so well served. In fact, in most places this is accomplished through home made scripts. Exceptions are UCSFImage4 [**REF] and Focus (Biayi et al 2017). As we will explain here, \scipionbox differs from these solutions in a larger flexibility both from the point of view of users and facility staff.  [**this parragraph need to be improved]
