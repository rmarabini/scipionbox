\begin{abstract}
  
Three dimensional electron microscopy (3DEM) is becoming a very data-intensive field in which vast amounts of experimental data are acquired at high speed. To manage such large-scale image data  projects, we have previously described  a modular workflow system called \scipion \citep{conesa2018}. We present here the first main extension of \scipion that allows to process data in streaming. Streaming processing allows to overlap data acquisition with the first steps of digital image processing and therefore helps to: (1) detect problems at early stages, (2) save computing time and (3) provide users with a in deep evaluation of the data quality before the data acquisition is finished. This idea of processing the data while being acquired has been implemented in different laboratories mainly using custom-made scripts. The advantages of our \scipion solution are: (1) flexibility: users can customize their workflows from a  large collection of preloaded algorithms; (2) repeatability and traceability: workflows are stored and can be re-executed later, either with the same data or a different data set; (3) monitorization: there are special utilities that constantly check how the execution of the algorithms is going and inform about possible anomalies; (4) reports: we have developed several GUIs that are updated periodically and produce a graphical summary (e.g, CTF defocus values, system load, etc). This summary may be  generated in HTML format that can be easily published though a public web-site to provide access for external users; (5) backup: both local and remote backup of the data and image processing is provided. At present, \scipion has been installed and is in production mode in three facilities. And it is being considered in several more.

\textbf{Keywords:} electron microscopy

\end{abstract}
