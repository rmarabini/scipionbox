\subsection{Future Developments}

Currently, \scipion includes the appropriate code (python wrappers) to talk to the integrated EM packages (Xmipp, Eman, RELION, ...) at the time of the release. This implies that an update of any of the integrated EM packages immediately after the \scipion release, won't be available until next \scipion release (we are aiming for one release a year). To decouple EM-package releases from \scipion releases, we are working on making the wrappers for EM packages  independent from \scipion. Plans for ensuring this independence include: wrapper installation through pip (\url{https://pip.pypa.io/en/stable/}) and the reimplementation of the wrappers as plug-ins that will be automatically detected and added to the application menu. 

Additionally, in collaboration with the Xmipp team, we are pushing the streaming workflow with the aim to obtain a first initial volume. We are also adding structure modeling capabilities, by integration of some functionality from programs such as Coot \citep{emsley2010:coot}, Refmac \citep{Murshudov1997:refmac}, etc. 

As part of the EOSCPilot project (\url{https://eoscpilot.eu/cryoem}), we are also improving  how \scipion exports workflows in order to ease reporting of the work done. The goal is to write a workflow file that fully describes the image processing steps, so that if the same input data is provided to the workflow the same results should be obtained (making the data more compliant with FAIR guidelines given by the \citet{eu2016:fair}). This file could go with the raw data acquired by CryoEM facilities and deposited at common EM databases like EMPIAR or EMDB \citep{Patwardhan2016:databasesEM}. To facilitate the visualization of the workflow file in any of these databases, we are also developing a  webcomponent (\scipion \emph{workflow viewer}) that will easily allow these repositories to visualize the workflow on their web pages (\url{https://github.com/I2PC/web-workflow-viewer}). Finally, we are implementing a workflow repository. This repository will contain ready-to-use workflows that support a range of use cases.
